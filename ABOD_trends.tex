%\section{Using income and consumption to assess recent trends in the level and distribution of household resources}\label{sec:trends}



%Overall, this section has shown that conclusions about the rate of growth in household resources, and the evolution of inequality and poverty, are sensitive, for most sub-periods, to the way that resources are measured. Furthermore, the differences in inequality and poverty trends assessed using income and consumption (or between measures that do or do not include imputed resources from housing) cannot be captured simply as an intercept shift. Holding constant the treatment of the imputed resources from housing, the most important of the different impressions one gets when assessing living standards with income and consumption are that resources have grown by more when assessed using income than consumption (particularly in the late 1980s and late 2000s), and that resources have grown in a more unequal way (mostly due to the late 1980s). We have also shown that including the imputed resources from housing increases the rate of growth of household resources - reflecting the growth over time in the ownership of housing, and a trend of having larger housing units - and reduces measures of inequality.


%\newpage



%\subsection{Summary}
%Overall, this section has shown that conclusions about the rate of growth in household resources, and the evolution of inequality and poverty, are sensitive, for most sub-periods, to the way that resources are measured. Furthermore, the differences in inequality and poverty trends assessed using income and consumption (or between measures that do or do not include imputed resources from housing) cannot be captured simply as an intercept shift. Holding constant the treatment of the imputed resources from housing, the most important of the different impressions one gets when assessing living standards with income and consumption are that resources have grown by more when assessed using income than consumption (particularly in the late 1980s and late 2000s), and that resources have grown in a more unequal way (mostly due to the late 1980s). We have also shown that including the imputed resources from housing increases the rate of growth of household resources - reflecting the growth over time in the ownership of housing, and a trend of having larger housing units - and reduces measures of inequality.



