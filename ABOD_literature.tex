\section{Related Literature}\label{sec:literature}

%[Mike notes: I have not worked on this recently and there are many papers to add. The idea is to first cite key works on "why use consumption", and then give a more thorough review of papers that directly compare the 2, then cite all the UK stuff, and then the odds and sods at the end.]

Our approach in this paper is not to argue that consumption is a better or worse measure of living standards than income, but to accept that consumption and income are often different measures, even for the same household, and to ask how this changes our impression of how the distribution of household resources in the UK has changed over time, and how it changes the composition of the poor. 

Several papers have asked similar questions for the US, with the main question being whether inequality in consumption in the US grew by as much as inequality in income over the past few decades. The general conclusion of the first papers in this literature found that consumption inequality had grown by less than income inequality through the 1980s and 1990s (see, for example, Krueger and Perri (2006) and citations within). More recent papers (e.g. Fisher et al. (2014), Aguiar and Bils (2015) and Attanasio et al. (2015)) have concluded that over a long period of time, the rise in inequality looks similar assessed on either measure (and this despite these last three papers not using the same measures of consumption).

 shows trends in inequality from 1984 to 2011 assessed with income and consumption; as does 

Measurement issues: there is also an ongoing debate about the best way to measure expenditure (and thus consumption). This is dominated by papers using US data and focusing on the specifics of the Consumer Expenditure (CE) survey instrument, which collects information on expenditure using both a diary (to collect expenditure over a 2 week period) and an interview, where households are asked to recall expenditure in various categories over the past month. For example, Parker et al. (2009) and Aguiar and Bils (2015) present evidence that the CE survey is, over time, increasingly under-measuring the expenditure of high-income households; Attanasio et al. (2007) compare the diary and interview components 
 
 For example, Attanasio et al (2006); Attanasio et al (2012); Bee et al (2014). 


The theoretical argument in favour of consumption has two related elements. First, because it is the consumption of goods and services, and the flow of benefits from durables and physical assets, that provides utility for households, a household's consumption in a given period will be a much more direct measure of the standard of living it is enjoying than its income. Second, because households can borrow or save, the amount of consumption in any period is not constrained to be equal to income in that period, and standard economic arguments suggest that consumption will better reflect expected lifetime resources than income (\cite{Poterba1989}, Cutler and Katz (1992) and Slesnick (1993); Blundell and Preston (1996)).

A practical argument for using consumption rather than income in developed countries, and particularly when examinng issues concerned with having low levels of resources rests on the claim that consumption is less likely to be mis-measured than income by standard household surveys  (Meyer and Sullivan 2003, 2004, 2008, 2011; Brewer, Etheridge and O'Dea, 2015) THESE NEED CHECKING TO FIND OUT WHICH ARE ABOUT MEASUREMENT ISSUES] and that survey estimates of household consumption are more strongly correlated with indicators of low material well-being than survey estimates of household income (Meyer and Sullivan (2003); Brewer and O'Dea (2010)).\footnote{Meyer and Sullivan's claims about the relative mis-measurement of income and expenditure have been called into question by, inter alia, Bavier (2008), and various authors have argued that expenditure data in the US is also measured with error: see, for example, Attanasio et al. (2005), Parker et al. (2009) and Aguiar and Bils (2011). MS (2011) contains a good guide to the debate on this issue, which we do not cover here as our interest lies in data from the UK, rather than US.}



Our motivation is similar to that of Meyer and Sullivan (2009), who comprensively assess the level and trend of poverty in the US using both income and consumption; it concludes that consumption poverty rates often indicate large declines, even in recent years when income poverty rates have risen and that the patterns are very different across family types, with consumption poverty falling much faster than income poverty since 1980 for the elderly, but more slowly for married couples with children. Overall, they conclude that the overall picture of the change in poverty is much more favorable using consumption measures than income measures.(p38) [We compare our findings to these in our concluding section.]
[Other studies].

Previous UK work on income and spending includes: for example, Attanasio et al. (2006) and Brewer et al. (2006) both directly compare measures of relative poverty based on expenditure (sic) and income; Goodman and Oldfield (2004) and Blundell and Etheridge (2010) directly compare inequality in consumption and income, and Carrera (2010) assesses how our impression of the redistributive nature of the tax and benefit system depends on whether one uses expenditure or income to rank households. \footnote{A parallel strand of the literature studies changes over time in the joint distribution of income and consumption to try to understand the relative importance of temporary and permanent shocks to income, including papers such as Krueger and Perri (2006) and Attanasio et al (2009) which use US data, and a series of papers by Blundell and co-authors (Blundell and Preston (1996, 1998), Blundell and Etheridge (2010) and Blundell et al (2011)), which use the same UK data as we do.} However, this paper extends and updates these by presenting a comprehensive assessment across all groups in society, and across four decades of micro-data; we also go to greater lengths than previous studies to construct consistent and comparable measures of consumption and income, and to adjust these correctly to account for changes in relative prices.

Our work is also related to other strands of the literature. First, researchers have assessed how our impression of the income distribution changes with different concepts of income. For example, Sutherland and Zantomio (2007) and Barnard et al. (2011) look at how the distribution of income in the UK alters when the value of public services is included, and Frazis and Stewart (2011) examine how inequality in the US changes when one adds a measure of home production to household income. We do, though,m consider housing, and in doing so we considerably extend Mullan et al (2011), which examines how the income distribution in the UK changes when one imputes income from housing but does this only for one year of data. Using Canadian data, Milligan (2008), shows how the well-being of elderly households relative to working-age households is very sensitive to whether one imputes a consumption flow from housing. Second, other work has assessed whether income gives the same impression of the level, composition and trends of who is poor as do measures of low living standards based on neither income nor spending, such as a measure of material deprivation or a hardship index.\footnote{ Definitions of these terms are not entirely standardised, but material deprivation is usually defined as an enforced lack of certain goods or access to certain services: see Mack and Lansley, 1985 for an early use of this, and Pantazis et al. 2006 for a recent one; and Boarini and d'Ercole (2006) for international experience and see McKay, 2004 for a critique.} For example, Bradshaw and Finch (2003) showed, using UK data, the lack of overlap between those who had a relative low income, and those who were defined as subjectively poor, or who had a high level of material deprivation, Calandrino (2003) found that the incidence of material deprivation amongst households in GB was lower in the bottom income decile group than the second income decile group, and Brewer et al. (2009) show that many of the children living in households with the very lowest incomes (first or second percentile of the overall income distribution) have lower levels of material deprivation than most other children in the bottom half of the income distribution.


%In the UK, there are four measures of child poverty defined in legislation, all in terms of a low household income (see \url{http://www.legislation.gov.uk/ukpga/2010/9/contents for details}, with similar targets existing at the level of the European Union (see Annex I of \url{http://ec.europa.eu/eu2020/pdf/council_conclusion_17_june_en.pdf}), and most policy and academic work in the UK continues to use income measures of poverty.
%\footnote{In earlier work also using US data, Sabelhaus and Groen (2000) had argued that the skewness of consumption-income ratios observed in the Consumer Expenditure Survey was impossible to rationalise given data on income variability and plausible specifications of how households choose levels of consumption.}
%These both provide data-driven reasons to prefer consumption over income when assessing the level of household resources (or living standards).
%\footnote{This needs to be moved or deleted.  Blundell and Preston (1996) highlight some difficulties with using comparisons of consumption levels to infer differences in lifetime resources, such as when comparing households at different stages of their lifecycle or when comparing individuals who are born many years apart.}



