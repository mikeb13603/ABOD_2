\section{Related Literature}\label{sec:literature}

%[Mike notes: The idea is to first cite key works on "why use consumption", and then give a more thorough review of papers that directly compare the 2, then cite all the UK stuff, and then the odds and sods at the end.]

Our approach in this paper is to accept that consumption and income are often different measures, even for the same household, and to ask how this changes our impression of how the distribution of household resources in the UK has changed over time, and how it changes the composition of the poor. \footnote{This is complementary to the strand of the literature that studies changes over time in the joint distribution of income and consumption to try to understand the relative importance of temporary and permanent shocks to income; this includes, for example, Krueger and Perri (2006), Attanasio et al (2009) and Blundell et al (2008) which use US data, and Blundell and Preston (1996, 1998), Blundell and Etheridge (2010) and Blundell et al (2013) for the UK.} 

Several papers have asked similar questions for the US, with the most-studied issue being whether inequality in consumption in the US has grown by as much as inequality in income over the past few decades. The general conclusion of the first papers in this literature was that consumption inequality had grown by less than income inequality through the 1980s and 1990s (see, for example, Krueger and Perri (2006) and citations within, Heathcote et al. (2010), Meyer and Sullivan (2013) [5 decades]), although more recently more papers (e.g. Fisher et al. (2014), Aguiar and Bils (2015) and Attanasio et al. (2015)) have concluded that over a long period of time, the rise in inequality looks similar assessed on either measure [(and this despite these last three papers not using the same measures of consumption)]. (Fisher et al. (2014) also argues that changing the period over which changes in inequality are measured can change the conclusions). We deliberately do not study data from the Great Recession, but Meyer and Sullivan (2013) and Fisher et al. (2014) both study inequality in the US through the Great Recession, and both find income inequality to have grown while consumption inequality fell. Rather fewer papers have looked at whether income and consumption give differing impressions of poverty in the US: the most comprehensive is Meyer and Sullivan (2012) [Brookings], which assesses trends in poverty in the US since the early 1960s (with earlier papers from the same authors (Meyer and Sullivan (2003, 2008) looking at lone mother households). Some of the findings of Meyer and Sullivan (2012) are about issues specific to the US, such as the use of and construction of certain price deflators, and the fact that the official US definition of poverty uses an unconventional definition of income. Having corrected for both these, though, Meyer and Sullivan (2012) find that poverty assessed using consumption fell by more than poverty assessed using income between 1980 and the late 2000s. Although Meyer and Sullivan (2012) do not look explicitly at the composition of the poor, which is the focus on our sections [X] and [X], they do show that consumption poverty fell by more than income poverty particularly for lone parents and the elderly, implying that shifting from income to consumption as the way to assess poverty would skew the composition of the poor away from these groups and towards working-age single adults and couples and to married couples with children.

[Need to look at this paragraph]. Previous UK work on income and spending/consumption includes: for example, Attanasio et al. (2006) and Brewer et al. (2006) both directly compare measures of relative poverty based on expenditure (sic) and income; Goodman and Oldfield (2004) and Blundell and Etheridge (2010) directly compare inequality in consumption and income. However, this paper extends and updates these by presenting a comprehensive assessment across all groups in society, and across four decades of micro-data; we also go to greater lengths than previous studies to construct consistent and comparable measures of consumption and income, and to adjust these correctly to account for changes in relative prices. and Carrera (2010) assesses how our impression of the redistributive nature of the tax and benefit system depends on whether one uses expenditure or income to rank households. We do, though, consider housing, and in doing so we considerably extend Mullan et al (2011), which examines how the income distribution in the UK changes when one imputes income from housing but does this only for one year of data. Using Canadian data, Milligan (2008), shows how the well-being of elderly households relative to working-age households is very sensitive to whether one imputes a consumption flow from housing. The mismatch at the bottom has been known for some time [...]
% Heathcote et al. (2010) is wage and consumption

%DOcuments read: Fisher and Smeeding RIW. 
% there are 2 Meyer and Sullivan (2013)s: the AERP&P piece, and the one on trends since the 1960s



%In the intro. The theoretical argument in favour of consumption has two related elements. First, because it is the consumption of goods and services, and the flow of benefits from durables and physical assets, that provides utility for households, a household's consumption in a given period will be a much more direct measure of the standard of living it is enjoying than its income. Second, because households can borrow or save, the amount of consumption in any period is not constrained to be equal to income in that period, and standard economic arguments suggest that consumption will better reflect expected lifetime resources than income (\cite{Poterba1989}, Cutler and Katz (1992) and Slesnick (1993); Blundell and Preston (1996)).

%A practical argument for using consumption rather than income in developed countries, and particularly when examinng issues concerned with having low levels of resources rests on the claim that consumption is less likely to be mis-measured than income by standard household surveys  (Meyer and Sullivan 2003, 2004, 2008, 2011; Brewer, Etheridge and O'Dea, 2015) THESE NEED CHECKING TO FIND OUT WHICH ARE ABOUT MEASUREMENT ISSUES] and that survey estimates of household consumption are more strongly correlated with indicators of low material well-being than survey estimates of household income (Meyer and Sullivan (2003); Brewer and O'Dea (2010)).\footnote{Meyer and Sullivan's claims about the relative mis-measurement of income and expenditure have been called into question by, inter alia, Bavier (2008), and various authors have argued that expenditure data in the US is also measured with error: see, for example, Attanasio et al. (2005), Parker et al. (2009) and Aguiar and Bils (2011). MS (2011) contains a good guide to the debate on this issue, which we do not cover here as our interest lies in data from the UK, rather than US.}


This paper does not try to argue that consumption is a better (or worse) measure of living standards than income, but other papers have focused on the relative accuracy of micro-data on household spending and income. In particular, with the same UK data as used in this paper, Brewer et al. (2016) argues that the size of the discrepancy between income and spending reported by those at the bottom of the income distribution in the UK is too great to be caused by consumption-smoothing alone (Sabelhaus and Groen (2000) make a similar argument for the US data), and they also show that the coverage rates of income from state cash benefits, and especially means-tested or income-related benefits, in household income surveys is low and falling (see Meyer et al (2015) [Meyer, Mok and Sullivan JEP] for an extremely detailed assessment of this issue for the main US household surveys, and Meyer and Mittag (2015) for an assessment of how correcting this under-reporting of income would affect estimates of the income-poverty rate). The conclusion to be drawn from the results in Brewer et al. (2016) is similar to the one made by Meyer and Sullivan (2003) from their analysis of equivalent US data: that reporting a low level of spending [or is it having a low consumption] in a household survey is a more accurate indicator that a household is in poverty than reporting a low level of income.  

However, there is also an ongoing debate about the accuracy of the data on expenditure traditionally collected from household surveys (and used to derive measures of household consumption). A substantial problem is that of a declining coverage rate, where coverage is defined as the implied economy-wide expenditure captured by the CE survey as a ratio of that in the National Accounts: see, for example, \citet{Barrettetal2014}. Parker et al. (2009) and Aguiar and Bils (2015) both present evidence that the CE survey is, over time, increasingly under-measuring the expenditure of high-income households, which has clear implications for the use of consumption data to measure inequality. Accordingly, Meyer and Sullivan (2013a,b) and Aguiar and Bils (2015) both offer a method of adjusting reported expenditure in the CE survey to correct for the under-reporting. (However, in the other direction, Bee et al. (2014) presents evidence that the coverage rate of the CE survey - the ratio of spending captured by the CE survey to equivalent categories of spending in national accounts - is both high and not falling for several important expenditure categories.) There is also a considerable focus on the specifics of the Consumer Expenditure (CE) survey instrument, which collects information on expenditure using both a diary (to collect expenditure over a 2 week period) and an interview, where households are asked to recall expenditure in various categories over the past month (see Carroll et al. (2015) for a comprehensive tour of the key issues). For example, Attanasio et al. (2007) and (2012) compare the diary and interview components of the CE survey, and conclude that data from the Interview component understates the rise in consumption inequality, but Bee et al. (2014) argues that using the Diary data alone is also likely to produce misleading assessment of inequality trends; accordingly, Browning and Crossley (2009) and Battistan and Padula (2016) suggest different ways of combining the information from the Diary and the Interview components to reduce measurement error. 

[Some of this is a cut and paste from BEOD]. These issues have been less thoroughly studied in the UK. As we explain in Section \ref{sec:measure}, data on household spending in the UK's LCFS/FES comes mainly from a diary, supplemented with information from an interview about spending on infrequently purchased items; unlike the CE survey, these diary and interview components provide complementary information and so there is little controversery over which to prefer or how best to combine them. Similar to the CE survey, the implied economy-wide expenditure captured by the LCFS is considerably lower than that reported in National Accounts (see \citealp{Deaton2005}, \citealp{Attanasio2006}, \citealp{Crossley2010} and \citet{Barrettetal2014}). We therefore might be concerned that expenditure data suffers from
its own under-reporting issues. The LCFS is the only nationally-representative
survey that collects comprehensive data on household budgets and,
as such, it is hard to validate household reports. However, \citet{LeicesterOldfield2009}
compare the expenditure patterns reported in the LCFS with those observed
in a commercial data set (the Kantar UK Worldpanel, previously known
as the TNS UK Worldpanel). The latter data covers spending on food
and groceries and is widely used in industry for market research.
\citet{LeicesterOldfield2009} find that, for each type of family,
levels of reported spending in the LCFS are \textit{higher} than those
in the Worldpanel.  

[Not sure where this goes. Possibly in the data section?]. The existing literature has used two different measures of consumption: non-durable consumption, and total consumption (which would include the consumption flow from owned durables, the most important being vehicles and property). A measure of consumption should exclude items of spending that are really investments, but there is not yet consistency over how to treat (for example) spending on medical care or education, although these are both a lot less important for our UK data than for data from the US. Our approach is to construct a measure of total consumption, and to compare that to a conceptually-equivalent measure of income that also includes the implicit income that accrues to those who own durables. Then, to isolate how significant is the inclusion of the consumption flow, or implicit income, from housing, we construct additional series of consumption and income that do not include the consumption or income from housing: this is therefore close to, but not the same as, non-durable consumption.

Our work is also related to other strands of the literature. First, researchers have assessed how our impression of the income distribution changes with different concepts of income. As described in Section \ref{sec:measure}, we address fully the implicit income accruing to home owners. Paulus et al. (2010) and Tonkin (2015) look at how the distribution of income in the UK alters when the value of public services is included, and Frazis and Stewart (2011) examine how inequality in the US changes when one adds a measure of home production to household income. These are both important sources of resources, but both are hard to estimate robustly (and especially over a long time-series of data), and omitting them both should not change the different impressions given by income and consumption. Second, other work has assessed whether income gives the same impression of the level, composition and trends of who is poor as do measures of low living standards that based on neither income nor spending, such as a measure of material deprivation or a hardship index.\footnote{ Definitions of these terms are not entirely standardised, but material deprivation is usually defined as an enforced lack of certain goods or access to certain services: see Boarini and d'Ercole (2006) for international experience and see McKay (2005) for a critique.} Brewer et al. (2009) shows that children living in households with the very lowest incomes (first or second percentile of the overall income distribution) have lower average levels of material deprivation than children in the whole of the bottom half of the income distribution; see also Belfield et al. (2015).


%In the UK, there are four measures of child poverty defined in legislation, all in terms of a low household income (see \url{http://www.legislation.gov.uk/ukpga/2010/9/contents for details}, with similar targets existing at the level of the European Union (see Annex I of \url{http://ec.europa.eu/eu2020/pdf/council_conclusion_17_june_en.pdf}), and most policy and academic work in the UK continues to use income measures of poverty.
%\footnote{In earlier work also using US data, Sabelhaus and Groen (2000) had argued that the skewness of consumption-income ratios observed in the Consumer Expenditure Survey was impossible to rationalise given data on income variability and plausible specifications of how households choose levels of consumption.}
%These both provide data-driven reasons to prefer consumption over income when assessing the level of household resources (or living standards).
%\footnote{This needs to be moved or deleted.  Blundell and Preston (1996) highlight some difficulties with using comparisons of consumption levels to infer differences in lifetime resources, such as when comparing households at different stages of their lifecycle or when comparing individuals who are born many years apart.}
