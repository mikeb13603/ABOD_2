\section{Conclusion}\label{sec:conclusion}

Blah blah. From introduction:

Our results can be grouped into four areas. 

First, we find that the distribution of income and consumption are quite different, in every year of our data. There was little difference between summary measures of inequality in income and consumption in the first half of the 1980s, but income inequality then grew considerably faster, where it remains less equally distributed than consumption. [Or: ``Inequality and relative poverty grew less rapidly when measured with consumption, partly because consumption at the bottom grew more strongly than income in the 1980s, and because consumption at the top grew less strongly than income in the 1990 and 2000s [Mike says: text following "`partly"' not all shown in this paper.]'']. A similar story is true for measures of relative poverty (having a household income, or consumption, below 60\% of the national median). [Resources have grown by more when assessed using income than consumption (particularly in the late 1980s and late 2000s).]  

Second, we present empirical evidence that the set of households with a low consumption are more likely to reflect households who have low permanent resources, as standard economic theory would suggest, than the set of households with a low income. Although this does not necessarily mean that policy-makers should target anti-poverty programmes on those with a low consumption, it does mean that one should look at consumption as well as/instead of income when assessing trends in poverty, and particularly when assessing which groups in society are at risk of poverty. [Mike says: is this the place to resurrect the DiD analysis that Cormac did, directly assessing how well consumption and income are at identifying the deprived?]. 

Third, at an aggregate level, we show that adding the implicit income or consumption from housing to measures of household resources makes a substantial difference to average annual growth rates in household living standards over the past 30 years, even after an appropriate correction to the price deflator [, and particular so for elderly households] [Mike says: last bit was in Brewer and O'Dea but is not currently shown in this paper], and also reduces the apparent growth in inequality in household resources. 

Fourth, at a micro-level, moving to a measure of resources that includes 
the implicit income or consumption from housing alters the characteristics of those deemed to be poor by giving less weight [wrong phrase] to those more likely to be temporarily poor (the unemployed and those with high education); it also changes dramatically the age-cohort profile of poverty, in a way that mirrors the large age-cohort profiles of home ownership, with the result that old age is rapidly losing its [ability to predict having a low living standard] [yuck].