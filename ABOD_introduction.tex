
\section{Introduction} 

%Point of this paper: comprehensive review of the trends in inequality, and in the level and composition of poverty that one gets with different measures of household resources. 

%Spare: In doing so, we pay particular attention to the role of the implicit rental income derived from, or the implicit consumption flow from, owner-occupied housing. 

Measuring poverty requires us to assess the living standards achieved by, or the resources available to, households, and there is a long debate over whether this is best done with income or consumption. On the one hand, there are conceptual reasons to prefer the use of consumption as a general measure of a household's living standards (\cite{Poterba1989}, Cutler and Katz (1992), Slesnick (1993)), as well as practical reasons to prefer the use of consumption to identify the poorest (Meyer and Sullivan 2003, 2004, 2011; Meyer, Kok and Sullivan, 2015; Brewer et al., 2016). However, it remains common in developed countries to use poverty lines defined with respect to household income. 

But the debate for or against the use of consumption rather than income is of little practical importance if trends in the level of distribution of living standards appear similar for both measures, and if the set of households with a low consumption were broadly similar to those with a low income. Using a long span of consistent survey-based micro-data on income and consumption for the same households, this paper assesses what impact the way we measure household resources has on our impression of trends in the composition of those in poverty in the UK. In doing this, we develop the findings of Brewer et al. (2016), which documents thoroughly the mis-match between reported income and reported spending for households with low resources in the UK and presents evidence that this is much more likely to be due to under-reporting of income than either over-reporting of spending or consumption-smoothing. We stress, though, that even if one takes a different view on the underlying causes, there remains a significant mismatch between measures of income and spending for households with low resources, and it is therefore important to know to what extent income and consumption give different impressions of trends in, and the composition of, households with low resources. 

Our broad approach is to analyse (and then compare) the trends in the level and distribution of households' resources, and the risk factors of having low resources, using both income and consumption as our measure of household resources, and using a long time-span of data covering [x thousand] households over three decades. As we explain in section \ref{sec:literature}, we improve on past work by focusing attention on the implicit rental income - or the imputed consumption flow - that accrues to homeowners. This is important for two reasons. First, some work which has compared the use of income and consumption to measure households' resources has compared a measure of disposable, cash, income with a measure of full consumption, including the consumption flows from durables like housing ([give citations: I think it's all of the recent US work]). This is an unfair comparison: it is just as sensible to think that consumer durables provide their owners with an implicit stream of rental income as it is to think that they provide an implicit stream of consumption (and we must think in this way if we work with the Haig-Simons concept that income less consumption equals change in net worth). Our approach is first to compare a usual measure of cash income with an equivalent measure of consumption (close to non-durable consumption), and then to assess what changes when we add to each of these the implicit rental income, or consumption flow, from housing. Second, given that all official and most academic analysis of household incomes and poverty in the UK uses an income concept that does not include the implicit rental income that accrues to owner-occupiers (and the partial availability of data on implicit rental income means that most international comparisons do likewise; see, e.g., Morelli et al (2015) [handbook chapter]), this paper presents the most comprehensive assessment of what impact including these implicit resources would have on poverty in the UK.    

%[Spare: Our approach is to construct a measure of total consumption, and to compare that to a conceptually-equivalent measure of income that also includes the implicit income that accrues to those who own durables. Then, to isolate how significant is the inclusion of the consumption flow, or implicit income, from housing, we construct additional series of consumption and income that do not include the consumption or income from housing: this is therefore close to, but not the same as, non-durable consumption.]

Our analysis therefore compares four measures of household resources, and examines trends in the composition of the poorest. Our results can be grouped into two areas. First, we present empirical evidence that the set of households with a low consumption are more likely to reflect households who have low permanent resources, as standard economic theory would suggest, than the set of households with a low income.  We do this by investigating directly the economic and demographic characteristics of households at the bottom of each of the distributions, but also by examining which of low income and low consumption is more strongly correlated with other measures of living standards.  Although this does not necessarily mean that policy-makers should target anti-poverty programmes on those with a low consumption, it does mean that one should look at consumption as well as/instead of income when assessing trends in poverty, and particularly when assessing which groups in society are at risk of poverty. Second, we show that moving to a measure of resources that includes the implicit income or consumption from housing alters the characteristics of those deemed to be poor by giving less weight [wrong phrase] to those more likely to be temporarily poor (the unemployed and those with high education) [work on this].

%First, we find that the distribution of income and consumption are quite different, in every year of our data. There was little difference between summary measures of inequality in income and consumption in the first half of the 1980s, but income inequality then grew considerably faster, and has since remained less equally distributed than consumption. This is because consumption at the bottom grew more strongly than income at the bottom in the 1980s, and because consumption at the top grew less strongly than income at the top in the 1990 and 2000s. A similar story is true for measures of relative poverty (having a household income, or consumption, below 60\% of the national median). [Resources have grown by more when assessed using income than consumption (particularly in the late 1980s and late 2000s).]  

%Second, adding the implicit resources from housing to measures of household resources makes a substantial difference to average annual growth rates in household living standards over the past 30 years, even after an appropriate correction to the price deflator [, and particular so for elderly households] [Mike notes: last bit was in Brewer and O'Dea but is not currently shown in this paper]. It also reduces the size of the rise in inequality in household resources.

The rest of the paper is arranged as follows.  Section \ref{sec:literature} discusses previous studies that have compared consumption and income as measures of household resources. Section \ref{sec:measure} discusses how we construct measures of consumption and income from 30 years worth of household surveys. Section \ref{sec:composition} examines how our impression of who is poor changes when we switch from using income to using consumption as the measure of resources. Section \ref{sec:housing} assesses how important is the treatment of implicit income from housing when constructing measures of household living standards. Section \ref{sec:conclusion} concludes. Appendices provide more detail on the underlying data sources, and supplementary results.
