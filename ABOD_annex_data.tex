\section{Further details about the measures of income and consumption}\label{data_annex}


%\subsection{The Living Costs and Food Survey}

%The response rate in the LCFS at the start of our data period was approximately 70\% and has shown a steady decline to 52\% in the most recent year of data (REFERENCE).  These response rates are slightly lower than in the Family Resources Survey  (62\% in the recent year of data (REFERENCE)) - the survey which is used to calculate the official measures of income poverty. However, this difference comes not from large differences in the proportion refusing to participate, but from a greater number of households in the intended sample who the survey team for the LCFS fail to make contact with.


\subsection{Income}
UK government assessments of income inequality and poverty are published in a document called Households Below Average Income (hereafter HBAI), and the name is also used to describe the associated micro-dataset (the name is misleading, as the micro-data and published statistics relate to the entire income distribution). The intent is that any reasonable household survey dataset with information on household composition and sources of income could be used to derive a measure of HBAI income. The official HBAI series is based on the LCFS and its predecessors until 1993/94, and on the FRS from 1994/95;  we have generated our own equivalent series based on the LCFS from 1994/95 in order to create a consistent series based on the LCFS and its predecessors. One difference between our measure and hthe official measures after 1994/95 is that the definition of income used in the official analyses of poverty further deducts contributions by parents to any children they have who are students living outside the household, but the LCFS data does not allow us to do this. One difference between  the measure of income before and after 1994/95 us that, from 1994/95,  payments into personal pensions and maintenance payments are deducted from the measure of income.

The HBAI document sets out the precise definition of income that government statisticians are seeking to measure, and the various methods that they use for constructing the HBAI micro-data). The key factors are as follows. First, the measure of income, described as ``net household disposable income'', comprises all forms of cash income plus a very few, government-provided, near-cash benefits-in-kind, less personal taxes paid (mostly based on self-reports, although some are imputed) less some transfers to other individuals and less some forms of saving. Income is measured at the household level, and equivalised for household size and composition.  Other than some small government-provided near-cash benefits-in-kind, no allowance is made for non-cash incomes such as those from housing or unrealised capital gains. This definition of income which we hereafter call ``HBAI income'' is known in the HBAI document as income ``before housing costs [are deducted]''; an alternative measure of income, known as income ``after housing costs [are deducted]'', subtracts spending on rent, mortgage interest and water charges from BHC income. The measure differs from cash income in that includes the imputed value of free school meals for households containing children who receive them; the cash value of a free TV licence for those elderly households who are entitled to it; housing benefit that is paid direct to the landlord (the value of which is therefore not included in a household's cash income) and excludes council tax payments, payments into personal pensions, maintenance payments to those in other households and student loan repayments.

%[Mike asks: do we want this?] Table X compares the distribution of income in the LCFS and FRS for all financial years since 1994/95; in 1994/95, the two datasets gave very similar estimates of the income distribution, but in recent years, the estimate from the LCFS has been higher than that from the FRS across the distribution. The estimated Gini coefficients from both surveys are very similar, though. /footnote{The estimated Gini coefficient for the FRS incorporates an adjustment to the incomes of approximately the richest 1\% of households which has not been done for the LCFS households: see DWP (2011) for details.}]

\subsection{Imputing the consumption flow or imputed income from housing and vehicles}
As we lack data on property values, we use the rental value of the property as a measure of the consumption value of living in that property. This is (clearly) observed in the data for those households who rent their property from a private landlord. But we do not observe a rental value for owner-occupiers, and, for tenants of social landlords, we observe a rent which will typically be less than the market rent. We therefore need to estimate the rent that owner-occupiers and social tenants would pay for their property if they rented it on the private market. Our approach imputes a rent for each property based on the geographical region, the number of rooms and the local taxation bill (There were three different local taxation regimes through the period covered by our data: rates (until 1988 in England and Wales, 1989 in Scotland), the Community Charge (between the abolition of rates and 1993) and council tax (from the abolition of the Community Charge to the present). Rates and council tax both varied (positively) with the value of the property, but the Community Charge did not.).  We take households who rent an unfurnished property privately in all years of data, and split them into three groups defined by the education of the head of household (those who left school at or before age 16, those who left at age 17 or 18 and those who left at or after the age of 19); this attempts to take account partially of the fact that those at different points in the permanent income distribution might have different quality of housing that cannot be captured by the data that we observe. For each group, we estimate a median regression of the log of rent on a quadratic in local tax payments interacted with a dummy for the local tax regime (we do not allow the imputed rent of households to vary with the Community Charge), indicators for government office region, indicators for the number of rooms in the property, and indicators for financial year.  For all households, we then calculate a measure of imputed (log) housing consumption as the prediction from this median regression plus a draw from the empirical distribution of the regression residuals (the draw for a particular household is a random draw from the sample comprising the residuals for all households surveyed in the same year and with the same education level). Brzozowski and Crossley  (2010) write that ``Imputed (or predicted) rents and service flows are typically not very variable (because they are based on a small number of measured characteristics of the stocks). Including them substantially reduces the variability of the consumption bundle.'' Our procedure does not suffer from this concern, as the  (conditional-on-observables) variability in our imputed measure is, by construction, identical to that in the observed data. On the other hand, our approach implicitly assumes that this unobserved component of housing quality is uncorrelated not only with the few observables but also with income and other components of consumption.

For vehicles, we assign each household the average expenditure on vehicles by those with the same number of cars and in the same decile group of non-durable expenditure. This expenditure will be taken over the positive values of those who have purchased a car in the previous 12 months and the zero values of those who
consume but have not purchased a vehicle in the previous 12 months.

We are not able to impute credibly the consumption flow from other durables, as we do not have a comprehensive record of other durables owned. Instead, we make assume that expenditure on other durables equals consumption. An alternative (and in our view less preferred) approach is to subtract spending on other durables, without adding back an estimated consumption flow. Taking this approach, however, would make very little difference to our measure of consumption, as the ratio of durable expenditure for which we cannot credibly impute associated consumption to our measure of total consumption has a mean (median) of only 5\% (2\%).   We do not remove from consumption spending on childcare, out-of-pocket medical expenses, or education expenses. This is mostly because,  as the UK has a free-at-the-point-of-use health service, and free education for children aged  5-18, we think that any out-of-pocket spending on these items is likely to be discretionary and thus more like consumption spending than investment spending. In any case, medical and education expenses are very low in the UK (certainly compared to the US). The argument that spending on childcare should be treated as an investment is perhaps a little stronger, but the spending on childcare has not been collected in a consistent manner across the four decades; our approach of leaving it as part of consumption at least prevents us from introducing inconsistencies over time.

We note that the way in which we have added an imputed income from housing to our measures of broad income and consumption is valid only if the markets for housing and other consumer durables for which an income is imputed) are frictionless (so that we can conclude that households have equalized their marginal utility across consumption choices). /footnote{ We are very grateful to Tom Crossley for this point.}. In the case of housing, there are clear transaction costs (certainly financial and arguably psychological) to moving house. But we also note that this issue remains un-acknowledged in many papers which routinely construct a measure of income including the imputed income from housing.

\subsection{Prices indices}
[CAN WE HAVE A TABLE SHOWING RPI, THE BHC HBAI DEFLATOR, AND OUR CONSTRUCTED INDEX FOR CONSUMPTION?]

\subsection{Periodicity of income and spending}
Both the measure of spending and income in the LCFS are and measured over relatively short periods (and, because of this, conventionally reported as expressed in weekly terms). As mentioned earlier,  spending on most items is collected through diaries which cover a fortnight, but this is supplemented with estimates of the weekly spend on infrequently-purchased items, which are based on respondents' total spend over a longer period and given in response to survey questions. The concept of income in the LCFS is ``usual weekly income'': this is typically based on participants' most recent wage or salary payments (and equivalent for the other income sources), but this is then replaced with the usual wage or salary payment if the last payment was deemed by the respondent to be unusual.  So, for workers paid every month or 4 weeks, the measure of earnings is effectively usual monthly/4-weekly earnings expressed as a weekly equivalent. For workers paid weekly, the measure of earnings is usual weekly earnings.  So both income and spending are measured over much shorter periods (as well as periods that are similar to each other), than in the main US data (which measures income over the previous year, and spending over the previous quarter), but it is not the case that income and spending are collected for the same period of calendar time, as occurs, for example, in the Canadian FAMEX/SHS surveys.
