\section{How important is the treatment of resources from housing in determining who is poor?}\label{sec:housing}

Section \ref{sec:trends} showed that, in aggregate, adding the imputed resources from housing to measures of household resources (whether income or consumption) increases the apparent growth in living standards over the past 30 years, and reduces the apparent level of inequality.   This section explores how adding the imputed resources from housing changes our impression of which households are in poverty, in a similar way to how Section \ref{sec:composition} explored how moving from income to consumption as our measure of household resources altered the composition of who is in poverty. 

%\subsection{title needed?}
\subsection{The composition of the ``poor IHC'' and the ``poor XHC''}\label{subsec:housing}

We begin with analysis that is analogous to that in Section \ref{subsec:composition}, but we are now comparing the different impression we get of who is poor when resources are measured with or without the imputed resources from housing.  

%Table \ref{table:multinom_incon} shows relative risk ratios of several demographic characteristics for the outcomes ``in the bottom decile group of both income and consumption resource distributions'', $r_{IC}$, ``in the bottom decile group of the income distribution but not the consumption distribution'', $r_{I}$, and ``in the bottom decile group of the consumption distribution but not the income distribution'', $r_{C}$, with ``not in the bottom decile group of either the income and consumption distributions'' as the reference category. \footnote{The results given here are for the sub-period 1999-2009. The results for earlier decades show the same qualitative story, and can be found in the Appendix \ref{sec:annex_results}.} The key parameter of interest, reported in the column titled $r_{I}-r_{C}$, is then the difference between the relative risk ratio of being ``in the bottom decile group of the income distribution but not the consumption distribution'' and ``in the bottom decile group of the consumption distribution but not the income distribution'', and we report the $\chi^{2}$ test statistic for this difference being zero. 

Table \ref{table:ahc_bhc_old} shows relative risk ratios of several demographic characteristics for the outcomes ``in the bottom decile group of resources with and without the imputed resources from housing'', $r_{IX}$, ``in the bottom decile group of resources with the imputed resources from housing'', $r_{I}$, and ``in the bottom decile group of resources without the imputed resources from housing'', $r_{X}$, with ``not in the bottom decile group of resources with or without the imputed resources from housing'' as the reference category; the separate panels consider income and consumption separately.\footnote{As in Table \ref{table:multinom_incon}, these results use data from 1999 to 2009; results for other decades can be found in Appendix \ref{sec:annex_results}}. The key parameter of interest, reported in the column titled $r_{I}-r_{X}$, is then the difference between the relative risk ratio of being ``in the bottom decile group of resources with imputed resources from housing'' and ``in the bottom decile group of resources without imputed resources from housing'': a positive value means that the characteristic is a stronger predictor of being in the bottom decile group of the distribution of resources with imputed resources from housing than without. As before, we report the $\chi^{2}$ statistic for this difference being zero. 

%[Mike says: am slightly confused about the tables. The first of these 2 tables has rather few covariates, and I think reflects our ``old'' specification, and can therefore be ignored. But we don't have the fully interacted version (interacting education, employment status and age) that we had in the previous section; I think we need that].
%
%[IGNORE THIS SPECIFICATION]. The left-hand panel of Table \ref{table:ahc_bhc_old} shows that having a medium level of education (rather than a high level), being aged 50 or more, being self-employed, and (especially) being workless are all (separately) stronger predictors of being in the bottom decile group of the distribution of income without imputed income from housing than with. In the other direction, being aged 30-40 (compared to age 40-50) [Mike says Abi: where are the cut-points? is it 30-39, 40-49 etc?], and (especially) having a low level of education are both stronger predictors of being in the bottom decile group of the distribution of income with imputed income from housing than without. This means that when we add the imputed income from housing, the bottom decile group then contains fewer households that are aged 50 or more, or self-employed or workless, and more who are aged 30-40, or employed, or with low levels of education.  The right-hand panel shows the results for adding the imputed consumption from from housing to the measure of consumption: the results are qualitatively very similar, although the extent to which adding resources from housing changes the age gradient is greater for consumption than it is for income. Overall, the results in Table \ref{table:ahc_bhc} indicate that adding the imputed resources from housing alters the characteristics of those deemed to be poor by giving less weight [wrong phrase] to those more likely to be temporarily poor (the unemployed and those with high education). Adding the imputed resources from housing also reduces the slope of the age profile of poverty.

The left-hand panel of Table \ref{table:ahc_bhc} shows that having a medium level of education (rather than a high level), being self-employed, and not living in a multi-family household (and especially living in a single person household) are all stronger predictors of being in the bottom decile group of the distribution of income without imputed income from housing than with. In the other direction, being aged 30-40 or over 60 (compared to age 40-50) [Mike says Abi: where are the cut-points? is it 30-39, 40-49 etc?], having a low level of education, and having children in the household, are all stronger predictors of being in the bottom decile group of the distribution of income with imputed income from housing than without. This means that when we add the imputed income from housing to household income, the bottom decile group then contains fewer households that are self-employed, or single-person households, and more who are aged 30-40, or 60 or over, or employed, or with dependent children, or with low levels of education.  The right-hand panel shows the results for adding the imputed consumption from from housing to the measure of consumption: the results are qualitatively very similar, although the extent to which adding resources from housing changes the education gradient and the age gradient are a little different for consumption.\footnote{When we add the imputed income from housing to household consumption, the bottom decile group then contains fewer households that have a medium level education [oops!], are workless, or single-person households, and more who are young, or employed, or with dependent children. [Mike says: the education story doesn't quite work here].} 

[Mike says: it is clear that there is something big going on in the interactions between the dummies for a single person household and the ``household contains a child'' dummy, and the age dummies (when we control for being a single person household and being a household with children, we lose a lot of the age profile, presumably because many old people live in single person households and none have children!].

Overall, the results in Table \ref{table:ahc_bhc} indicate that adding the imputed resources from housing alters the characteristics of those deemed to be poor by giving less weight [wrong phrase] to those more likely to be temporarily poor (the unemployed and those with high education). Adding the imputed resources from housing also reduces the slope of the age profile of poverty. [Mike says: I don't think we have quite nailed these yet].

%
%\begin{sidewaystable}
%\caption{Demographics and the Bottom Decile of IHC \& XHC Distributions, 1999-2009: Relative Risk Ratios}
%\centering
%\begin{tabular}{l|cccc|cccc}
%\hline\hline 
	%& \multicolumn{4}{c}{\textbf{Income Bottom Decile}} &  \multicolumn{4}{c}{\textbf{Cons. Bottom Decile}} \\
	%&	$r_{IX}$	&	$r_{I}$	&	$r_{X}$ &	$r_{I}$-$r_{X}$&	$r_{IX}$	&	$r_{I}$	&	$r_{X}$	&	$r_{I}$-$r_{X}$\\
  %& se & se & se  & $\chi^{2}$ & se & se & se & $\chi^{2}$ \\
%\hline
%Left school $\leq$ 16	&	       1.37***  	&	       2.51***	&	      1.45***	&	1.06****	&	     					  3.17***	&	       2.89***	&	2.79***	&	0.11	\\
                   		 	%&	       0.10  	&	0.31	&	0.09	&	18	&	      
						 %0.23   	&	0.33	&	0.27	&	0.08	\\
%16 $<$ Left school $<=$ 19	&	       0.96   	&	       0.97  	&	1.46***	&	-0.48***	&	
				       %1.08   	&	       1.01  	&	       1.78***  	&	-0.78***	\\
                    		%&	       0.08   	&	0.09	&	0.14	&	7.8	&	     
					  %0.08   	&	0.09	&	0.18	&	24	\\
%Age $<$ 30	&	       1.92*** 	&	       1.89***  	&	1.95***	&	-0.06	&	
			       %2.07***  	&	1.30***	&	1.84***	&	-0.54***	\\
                    	%&	       0.13   	&	0.10	&	0.23	&	0.04	&	  
			     %0.17   	&	0.10	&	0.18	&	17	\\
%Age 30-40	&	      1.20***   	&	       1.44***	&	       0.99 	&	0.45***	&	
			       %1.34***   	&	       1.26***	&	1.08	&	0.18***	\\
                    	%&	       0.05   	&	0.10	&	0.09	&	9.5	&	  
			     %0.07   	&	0.06	&	0.06	&	3.7	\\
%Age 50-60	&	       0.63***	&	       0.36***	&	0.84***	&	-0.48***	&
			       %0.56***	&	       0.43***	&	       0.96 	&	-0.54***	\\
                    	%&	       0.02   	&	0.07	&	0.05	&	15	&	
			       %0.03   	&	0.03	&	0.09	&	54	\\
%Age 60-70	&	       0.16***	&	       0.17***	&	       0.27***	&	-0.10***	&	       								0.22***	&	       0.23***	&	0.53	&	-0.31***	\\
                    	%&	       0.01   	&	0.02	&	0.02	&	8.4	&	
			       %0.02   	&	0.02	&	0.04	&	97	\\
%Age $\geq$ 70	&	       0.09***	&	       0.15***	&	       0.25***	&	-0.10***	&	    					   		0.51***	&	       0.30***	&	       1.15 	&	-0.85***	\\
                    	%&	       0.01   	&	0.02	&	0.03	&	7.5	&	    
				   %0.05  	&	0.03	&	0.12	&	225	\\
%Workless	&	       14.7***	&	       11.7***	&	       16.1***	&	-4.47***	&	       					11.9***	&	      5.20***	&	       8.04***	&	-2.85***	\\
	%&	       0.81   	&	0.91	&	1.05	&	8.1	&	
		       %0.67   	&	0.50	&	0.28	&	31	\\
%Self Employed	&	       3.74***	&	      1.84***	&	       2.82***	&	-0.98**	&	       							0.80***	&	       0.97&	       0.73***	&	-0.24***	\\
	%&	       0.34   	&	0.22	&	0.25	&	5.5	&	
		       %0.07   	&	0.09	&	0.08	&	7.3	\\
%Constant            	&	       0.027**	&	       0.006***	&	0.010	&		&	
				       %0.012***	&	       0.010***	&	       0.008***	&		\\
                    	%&	       0.002   	&	0.001	&	       0.001 	&		&	 
			      %0.001   	&	0.001	&	0.001	&		\\
%\hline\hline
%\multicolumn{9}{l}{Significantly different from zero at the 10\% ($\star$),  5\% ($\star\star$) and 1\% level ($\star\star\star$).} \\
%\multicolumn{9}{l}{Omitted variables: Left school over 19 years old, household head aged between 40 and 50 years, employed. } 
%\end{tabular}
%\label{table:ahc_bhc_old}
%\end{sidewaystable}

\begin{sidewaystable}
\caption{Demographics and the Bottom Decile of IHC \& XHC Distributions, 1999-2009: Relative Risk Ratios}
\centering
\begin{tabular}{l|cccc|cccc}
\hline\hline 
	& \multicolumn{4}{c}{\textbf{Income Bottom Decile}} &  \multicolumn{4}{c}{\textbf{Cons. Bottom Decile}} \\
	&	$r_{IX}$	&	$r_{I}$	&	$r_{X}$ &	$r_{I}$-$r_{X}$&	$r_{IX}$	&	$r_{I}$	&	$r_{X}$	&	$r_{I}$-$r_{X}$\\
  & se & se & se  & $\chi^{2}$ & se & se & se & $\chi^{2}$ \\
\hline
Left school $\leq$ 16	&	       1.34***  	&	       2.09***	&	      1.49***	&	0.60**	&	     					  2.70***	&	       2.54***	&	2.85***	&	-0.32	\\
                   		 	&	       0.10  	&	0.26	&	0.11	&	6.0	&	      
						 0.23   	&	0.28	&	0.30	&	0.64	\\
16 $<$ Left school $<=$ 19	&	       1.00   	&	       0.85*  	&	1.71***	&	-0.58***	&	
				       0.83   	&	       0.92  	&	       1.78***  	&	-0.79***	\\
                    		&	       0.08   	&	0.08	&	0.14	&	12	&	     
					  0.06   	&	0.08	&	0.17	&	31	\\
Age $<$ 30	&	       1.79*** 	&	       1.95***  	&	1.61***	&	0.34	&	
			       1.83***  	&	1.57***	&	1.51***	&	0.07	\\
                    	&	       0.13   	&	0.19	&	0.18	&	1.4	&	  
			     0.178  	&	0.15	&	0.15	&	0.23	\\
Age 30-40	&	      1.14***   	&	       1.31***	&	       0.91	&	0.40***	&	
			       1.21***   	&	       1.28***	&	1.01	&	0.27***	\\
                    	&	       0.05   	&	0.10	&	0.09	&	8.2	&	  
			     0.06 	&	0.06	&	0.07	&	9.4	\\
Age 50-60	&	       0.71***	&	       0.73	&	0.81***	&	-0.08	&
			       0.78***	&	       0.74***	&	       0.95 	&	-0.21**	\\
                    	&	       0.03   	&	0.16	&	0.07	&	0.1	&	
			       0.05   	&	0.05	&	0.09	&	4.9	\\
Age 60-70	&	       0.19***	&	       0.57***	&	       0.23***	&	0.34***	&	       								0.37***	&	       0.60***	&	0.50***	& 0.10**	\\
                    	&	       0.02   	&	0.07	&	0.02	&	19	&	
			       0.03   	&	0.04	&	0.09	&	4.7	\\
Age $\geq$ 70	&	       0.10***	&	       0.54***	&	       0.17***	&	0.37***	&	    					   		0.79**	&	       0.86***	&	      0.90 	&	-0.04	\\
                    	&	       0.01   	&	0.06	&	0.02	&	30	&	    
				   0.08  	&	0.07	&	0.12	&	0.47	\\
Workless	&	       12.5***	&	       10.3***	&	       11.9***	&	-1.64	&	       									9.42***	&	      5.16***	&	       5.35***	&	-0.20***	\\
	&	       0.69   	&	0.89	&	0.73	&	1.4	&	
		       0.51   	&	0.50	&	0.19	&	0.15	\\
Self Employed	&	       3.88***	&	      1.69***	&	       3.41***	&	-1.72***	&	       							0.81**	&	       0.90 &	       0.89	&	0.01	\\
	&	       0.35   	&	0.21	&	0.31	&	14	&	
		       0.07   	&	0.09	&	0.09	&	0.03	\\
Couple	&	       0.87	&	      0.68***	&	       1.53***	&	-0.84***	&	       							0.68***	&	       0.58*** &	       1.09	&	-0.52***	\\
	&	       0.07  	&	0.08	&	0.16	&	40	&	
		       0.05   	&	0.05	&	0.17	&	14	\\
Single	&	       1.66***	&	      0.92	&	       6.46***	&	-5.55***	&	       							1.59***	&	       0.60***&	      6.66*** &	-6.05***	\\
	&	       0.14   	&	0.12	&	0.71	&	297	&	
		       0.13   	&	0.04	&	1.04	&	239	\\
Child Dummy	&	       1.27***	&	      5.36***	&	      0.72***	&	4.64***	&	       							2.09***	&	       3.80*** &	       0.77***	&	3.04***	\\
	&	       0.09   	&	0.59	&	0.06	&	289	&	
		       0.10   	&	0.22	&	0.06	&	378	\\
Constant            	&	       0.023**	&	       0.003***	&	0.004***	&		&	
				       0.010***	&	       0.006***	&	       0.004***	&		\\
                    	&	       0.002   	&	0.001	&	       0.001 	&		&	 
			      0.001   	&	0.001	&	0.001	&		\\
\hline\hline
\multicolumn{9}{l}{Significantly different from zero at the 5\% ($\star\star$) and 1\% level ($\star\star\star$).} \\
\multicolumn{9}{l}{Omitted variables: Left school over 19 years old, household head aged between 40 and 50 years, employed. } 
\end{tabular}
\label{table:ahc_bhc}
\end{sidewaystable}


[NEW TEXT BASED ON NEW TABLES NOT CONTAINED IN THIS DOCUMENT. 25 FEB.  VERSION WITH SINGLE INTERACTED WITH OVER 60: The left-hand panel of Table XXX shows that being self-employed or not in employment, and living in a single person household (especially if aged under 60) are all stronger predictors of being in the bottom decile group of the distribution of income without imputed income from housing than with.  In the other direction, being aged under 40 or over 60, having a high or (especially) low (compared with medium) level of education, living in a multi-adult household, and having children in the household, are all stronger predictors of being in the bottom decile group of the distribution of income with imputed income from housing than without. [THESE ARE HARD TO INTERPRET!]. This means that when we add the imputed income from housing to household income, the bottom decile group then contains fewer households that are self-employed or workless, fewer single-person households, especially those aged under 60, and more who are aged under 40, or over 60, or with dependent children, or in multi-adult families, or with low or high levels of education.  The right-hand panel shows the results for adding the imputed consumption from from housing to the measure of consumption: the results are a little different: living in a single person household, and being aged 50-59 are stronger predictors of being in the bottom decile group of the distribution of income without imputed income from housing than with; having a high or (especially) low (compared with medium) level of education,  living in a multi-adult household, having children in the household, and being aged 30-39 are all stronger predictors of being in the bottom decile group of the distribution of income with imputed income from housing than without. [THESE ARE HARD TO INTERPRET!]. This means that when we add the imputed consumption from housing to household consumption, the bottom decile group then contains fewer single-person households and those aged 50-59, and more who are aged 30-39, or with dependent children, or in multi-adult families, or with low or high levels of education.  

NEW TEXT BASED ON NEW TABLES NOT CONTAINED IN THIS DOCUMENT. 25 FEB.  VERSION WITH SINGLE INTERACTED WITH OVER 60 AND WITH EMPLOYMENT STATUS INTERACTED WITH AGE AND EDUCATION: Table XXX explores this conclusion further by interacting employment status with both age and education. As in Table XXX, the $r_{I}-r_{X}$ column of the left-hand panel of Table XXX shows that being self-employed, and living in a single person household (especially if aged under 60) are all stronger predictors of being in the bottom decile group of the distribution of income without imputed income from housing than with, and that being aged under 40, having a low level of education, living in a multi-adult household, and having children in the household are all stronger predictors of being in the bottom decile group of the distribution of income with imputed income from housing than without. What the interactions tell us is that being out of work and aged under 30 is a stronger predictor of being in poverty without imputed income from housing than with, and being out of work with a high level of education is a stronger predictor of being in poverty with imputed income from housing  than without [WHOOPS! FOR THE LAST ONE]. [AM STILL FIZZLING OUT HERE]. As before, the results are a little different in the right-hand panel, which shows what happens when we add the imputed consumption from housing to the measure of consumption: doing so means that the bottom decile group contains fewer more low-educated households, more households with children, more multi-adult households, but fewer single adult households. However, what is harder to explain is that it also contains more high-educated and non-working households. 




 
