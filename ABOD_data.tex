\section{Income and consumption in the UK: data and measurement}\label{sec:measure}

Our main dataset is the Living Costs and Food Survey (known between 2001 and 2007 as the Expenditure and Food Survey, and the Family Expenditure Survey before that; we refer to it as the LCFS), the only UK survey which contains comprehensive measures of both income and expenditure.

The LCFS is an annual, nationally-representative, cross-sectional survey with an annual sample of between 5,000 and 7,000 households. Its primary purpose is to provide data for the calculation of the commodity  weights for the UK's price indices. It aims to collect a comprehensive measure of household spending with a two-week diary, in which respondents are asked to record everything they  purchase, supplemented by a questionnaire in which respondents are asked about any spending on infrequently purchased items over the past number of months (for example, respondents are asked to record any spending on motor vehicles in the past 12 months, but any spending on household fuel in the past 3 months). It also collects data on household income measured over a relatively short period, as is standard in UK household surveys. We use data from 1978 to 2009.

\subsection{Consumption}
The existing literature has used two different measures of consumption: non-durable consumption, and total consumption (which includes the consumption flow from owned durables, the most important of which are vehicles and property). We use two measures of consumption. Both begin with a measure of household expenditure that simply records all spending by a household in a given period; we use, where available, responses to the questionnaire about spending on infrequently purchased items instead of information from the diary  [MIKE ASKS Cormac: DOES THIS NEED EXPANDING HERE OR IN THE ANNEX? WHAT ABOUT BENEFITS IN KIND LIKE FREE SCHOOL MEALS AND FREE TV LICENSES? Are these in expenditure? consumption? or income?]. 

To derive a measure of full consumption, we begin with this measure of expenditure, subtract spending on vehicles and housing (i.e. mortgage interest payments, capital payments and rent), viewing these outlays as investments in durables), and, as described in Appendix \ref{data_annex}, add an imputed consumption value for these two items.\footnote{A measure of consumption should exclude items of spending that are in fact investments, but there is not yet consistency in the literature over how to treat (for example) spending on medical care or education; we do not exclude spending on such items from our measure, but, with a universal health care system that is free at the point use, these a lot less important for our UK data than for data from the US.} Our second measure is identical except that it does not include the imputed consumption flow from housing: this is therefore close to, but not the same as, non-durable consumption, but we create this measure to isolate how significant is the inclusion of the consumption flow from housing. 


\subsection{Income}
We use two different measures of income in this paper.  Our first is the usual measures of HBAI income before housing costs (see Appendix \ref{data_annex} and DWP (XXXX)). The second is a broader measure of income intended to be directly comparable with our measure of consumption. To derive this, we start with the HBAI BHC income measure and subtract payments made to students from the Student Loan Company (these are loans, but, for reasons unclear to us, are treated like income in the HBAI income measure), and then add an estimate of the consumption flow from housing and motor vehicles net of the cash payments made on the same. The second of these adjustments takes account of the fact that ownership of a particular durable can be considered to yield an imputed flow of income just as it can be considered to yield an flow of consumption benefits, and it means that we make exactly the same adjustments to the income measure that we make when moving from expenditure to consumption./footnote{This partly addresses the concern in \cite{Bavier2008} about some of the arguments made by MS. Bavier argues that one should not compare consumption only to the measure of income used in the official analysis of poverty but to the best measure of income that can be derived. As the derivation of a consumption measure typically starts with expenditure data and makes adjustments in keeping with theoretical and empirical evidence about how best that data can be used to predict deprivation, then the odds are stacked against income predicting living standards better than consumption unless a similar process is carried out to the income data.}) 

\subsection{Adjusting for price changes and household composition}
We express all financial values in 2009 pounds, and use price indices based on the RPI to achieve this. \footnote{[PLEASE COULD CORMAC REVIEW THE ACCURACY AND NEED FOR THIS FOOTNOTE?] MS (2009) pay particular attention to how the choice of deflator materially affects conclusions about trends in living standards towards the bottom of the distribution. The UK has two main official measures of price inflation: the Retail Prices Index and the Consumers Prices Index: these differ in their formula and the coverage (for a summary of the differences, see Office for National Statistics (2011)). There are a number of reasons for our use of the RPI and variants thereof rather than the CPI for our price adjustments. These include the fact that it has been existence for the entirety of the period we consider (unlike the CPI); the fact that its coverage is broader (in particular it includes housing, which is omitted from the CPI), and the fact that the official poverty analyses produced by the UK government still (at time of writing) use the RPI rather than the CPI.}We do not use the actual RPI to deflate all of our measures of household resources, but instead make slight adjustments to reflect that our different measures of income, spending and consumption are constructed in different ways and therefore measure different things. In particular, we deflate measures of 
%cash income and 
cash expenditure using the RPI, we deflate measures of HBAI income using an official variant of the RPI which does not include the level of local taxation or housing depreciation in the basket of goods, and we deflate measures of consumption and income that include imputed housing income or consumption with our own variant of the RPI which does not include mortgage interest payments in the basket, and instead does include rent, but weighted in keeping with the budget share of imputed rent, rather than actual spending on rent. [Mike asks Cormac: should we make more of this? I thought you said years ago that this was important and noone else did it?]

We construct measures of income and consumption at the level of the household, and then adjust for household composition using the modified OECD scale; this means we assume the same equivalence scale is applicable for our different measures of household resources. Following usual UK practice, we re-base so that a two-adult household has a weight of 1, meaning that the scale becomes 0.67 for a single adult, 0.33 for each extra adult or child aged 14 or more, and 0.2 for every child aged under 14. We conduct all analysis at the level of the individual (including children), having assigned to each individual their household's equivalised household income or consumption; this is numerically equivalent to having household-level data on equivalised household resources and weighting all analyses by the number of people in the household.

